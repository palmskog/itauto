% easychair.tex,v 3.5 2017/03/15

\documentclass{easychair}
%\documentclass[EPiC]{easychair}
%\documentclass[EPiCempty]{easychair}
%\documentclass[debug]{easychair}
%\documentclass[verbose]{easychair}
%\documentclass[notimes]{easychair}
%\documentclass[withtimes]{easychair}
%\documentclass[a4paper]{easychair}
%\documentclass[letterpaper]{easychair}
\usepackage{fullpage}
\usepackage[]{minted}
\usepackage{amsmath,amsthm}


\newmintinline[icoq]{coq}{}
\setminted[coq]{escapeinside=\#\#,mathescape=true} %,fontsize=\footnotesize}
\newtheorem{example}{Example}

%
\title{À la Nelson-Oppen Combination for \icoq{congruence}, \icoq{lia} and \icoq{lra}}
%
\author{ Frédéric Besson}

% Institutes for affiliations are also joined by \and,
\institute{Inria, Rennes, France}

\authorrunning{F. Besson}

\titlerunning{À la Nelson-Oppen Combination for \icoq{congruence}, \icoq{lia} and \icoq{lra}}

\begin{document}

\maketitle

\begin{abstract}
  We propose a tactic for combining decision procedures using a
  black-box Nelson-Oppen scheme.
  %
  The tactic is instantiated for \icoq{congruence} and either \icoq{lia} or \icoq{lra}.
  %
  The development is available at \url{https://gitlab.inria.fr/fbesson/itauto}.
\end{abstract}


\section{Introduction}
\label{sect:introduction}

The Coq proof-assistant provides decision procedures for various logic
fragments. In practice, most of the goals do not fall in those
restricted fragments and, in that case, an interactive proof is
required.
%
However, there is sometimes a sweet spot when the goal can be solved by a
collaboration of decision procedures.
%
For instance, \icoq{intuition tac} enhances the expressive power of a
tactic \icoq{tac} by providing support for propositional logic.  Our
recent \icoq{itauto tac}~\cite{Itauto} shares the same goal but aims at improving the
completeness and efficiency of the combination.
%

Unfortunately, there is currently no support for solving goals that are
expressed in the combined decidable logic fragments of EUF~\cite{Ackermann} (Equality Logic with
Uninterpreted Functions) and LIA~\cite{presburger} (Linear Integer Arithmetic).
%
Yet, \icoq{congruence}\footnote{\url{https://coq.inria.fr/refman/proof-engine/tactics.html\#coq:tacn.congruence}}~\cite{Corbineau06} subsumes EUF and \icoq{lia}\footnote{\url{https://coq.inria.fr/refman/addendum/micromega.html\#coq:tacn.lia}}~\cite{BessonCP11,Besson06} solves LIA.
Moreover, Nelson and Oppen~\cite{NelsonO79} propose a combination scheme which is complete for the combination EUF+LIA.

In the following, we present our \icoq{smt}
tactic\footnote{\url{https://gitlab.inria.fr/fbesson/itauto}} which implements the Nelson-Oppen combination scheme in a black-box manner.

\section{Motivating Example}
The crux of the Nelson-Oppen scheme is that equality sharing is
sufficient\footnote{Under technical conditions that are not detailed here} for
a complete combination of two decidable theories  when
the unique shared symbol is equality. The following example
illustrates a somewhat painful interactive proof that is automated by our
\icoq{smt} tactic.
\begin{example}
  \label{exa:motivating}
  Consider the following goal.
\begin{minted}{coq}
Goal #$\forall$# (x y: Z) (P:Z -> Prop), x :: nil = y + 1 :: nil -> P (x - y) -> P 1.
\end{minted}
Neither \icoq{congruence} nor \icoq{lia} solves the goal. Yet, it can
be solved by only asserting equalities that are solved by either \icoq{congruence} or \icoq{lia}.
This is illustrated by the following proof script.
\begin{minted}{coq}
Proof. intros. assert (x = y+1) by congruence. assert (x-y = 1) by lia. congruence. Qed.
\end{minted}
\end{example}

\section{Nelson-Oppen Algorithm}
The first task of the \icoq{smt} tactic is to identify equalities that
are worth being propagated. This is done by the so-called
\emph{purification} phase. The second task consists in propagating
equalities until the goal is solved. These two phases are implemented
as an OCaml plugin.

\paragraph{Purification}
The \emph{purification} introduces fresh variables and
equations so that every term belongs to one and only one theory.
For Example~\ref{exa:motivating}, we would obtain the following goal.
\begin{minted}{coq}
  hpr1 : 1 = pr1,  hpr3 : y + pr1 = pr3,  hpr2 : x - y = pr2
  H  : x :: nil = pr3 :: nil,  H0 : P pr2
  ==========================================================
  P pr1
\end{minted}
The set of potential equations is then defined as
$
\{ x = y \mid (x,y) \in \mathit{Var} \times \mathit{Var} \}
$
where $\mathit{Var} = \{\icoq{pr1},\icoq{pr2}, \icoq{pr3}, \icoq{x} \}$.
%
The set of variables contains fresh variables but also the existing
variables that are at the interface between the two theories.  Here,
the variable \icoq{x} is an arithmetic variable using as argument of
the constructor \icoq{::}.

\paragraph{Theory Description} In order to perform the purification
phase, it is necessary to have the signature of the theory that is combined with EUF
\emph{i.e.}, the set of arithmetic types and operators.
%
This is done by declaring instances of the two following type-classes.
\begin{minted}{coq}
Class TheoryType(Tid:Type)(T:Type):Type. Class TheorySig(Tid:Type){T:Type}(Op:T):Type.
\end{minted}
Note that the type-classes are parametrised by an uninterpreted type
\icoq{Tid} that only used to identify a theory. In our case, \icoq{ZarithThy} is associated to
\icoq{lia} and \icoq{RarithThy} is associated to \icoq{lra}.

\paragraph{Equality Sharing} Our current Nelson-Oppen tactic is binary and can combine
\icoq{congruence} with either \icoq{lia} or \icoq{lra}.
%
After purification, we recursively try to prove one of the equality
using either \icoq{congruence} or the arithmetic tactic.  If tactic
$T_1$ succeeds at asserting an equation, we try to solve the goal
using tactic $T_2$. If $T_2$ fails, we iterate the process until none
of the equation can be proved.
%
As there is a quadratic number of possible equations, the combination
requires, in the worst case, a cubic number of calls to the decision
procedures.

\section{Conclusion and Limitations}
Our \icoq{smt} tactic improves automation and has the advantage of
reusing the existing tactics \icoq{congruence}, \icoq{lia} and \icoq{lra} in a
black-box manner. More experiments are needed to assess to what extent 
automation is increased in practice and whether efficiency is satisfactory.
%
Improving efficiency would require to adapt the decision procedures so
that they either prove a goal or assert equations.
%
Though this might not be a problem in practice, our implementation is
not complete. LIA is a so-called \emph{non-convex} theory which
requires propagating not only equalities but disjunctions of
equalities. Moreover, \icoq{lia} is first running \icoq{zify}. It is
currently unclear how this impacts the Nelson-Oppen scheme. 

\label{sect:bib}
\bibliographystyle{plain}
%\bibliographystyle{alpha}
%\bibliographystyle{unsrt}
%\bibliographystyle{abbrv}
\bibliography{biblio}

\end{document}

%%% Local Variables:
%%% mode: latex
%%% TeX-engine: default
%%% TeX-master: t 
%%% TeX-command-extra-options: "-shell-escape"
%%% mode: flyspell
%%% ispell-local-dictionary: "british"
%%% End: